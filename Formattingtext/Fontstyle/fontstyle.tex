\documentclass[12pt]{article}
\usepackage{amsmath}
\usepackage[none]{hyphenat}
\usepackage[a4paper, margin=1.5in]{geometry}
\title{Font Style}
\author{Navya Dasari}
\date{\today}
\usepackage{lipsum}

% textbf{}
% texttit{}
% emph{}
% underline{}
% textsc{}
% textrm{}
% textsf{}
% textttt{}
% textsuperscript{}
% textsubscript{}


\begin{document}
	
	\maketitle
	
	\paragraph{}
	
	An ecosystem can be visualised as a functional unit of nature, where living 
	organisms interact among themselves and also with the surrounding physical 
	environment. Ecosystem varies greatly in size from a small pond to a large forest or a 
	sea. Many ecologists regard the entire biosphere as a global ecosystem, as a composite 
	of all local ecosystems on Earth. Since this system is too much big and complex to be 
	studied at one time, it is convenient to divide it into two basic categories, namely the 
	terrestrial and the aquatic. Forest, grassland and desert are some examples of terrestrial 
	ecosystems; pond, lake, wetland, river and estuary are some examples of aquatic 
	ecosystems. Crop fields and an aquarium may also be considered as man-made 
	ecosystems. \\[12pt]
	
	
	An \textbf{ecosystem} can be visualised as a \textit{functional unit of nature}, where living 
	organisms interact among themselves and also with the surrounding physical 
	environment. \emph{Ecosystem varies greatly in size from a small pond to a large forest or a 
		sea}. Many \underline{ecologists} regard the entire biosphere as a global ecosystem, as a composite 
	of all local ecosystems on Earth. Since this system is too much big and complex to be 
	studied at one time, it is convenient to divide it into \textsc{two} basic categories, namely the 
	\textrm{terrestrial} and the \textsf{aquatic}. Forest, grassland and desert are some examples of terrestrial 
	\textsuperscript{ecosystems}; pond, lake, wetland, river and estuary are some examples of aquatic 
	\textsubscript{ecosystems}. Crop fields and an aquarium may also be considered as man-made 
	ecosystems.
\end{document}