\documentclass[12pt]{article}
\usepackage[none]{hyphenat}
\usepackage[a4paper, margin=1.5in]{geometry}
\title{Font Style}
\author{Navya Dasari}
\date{\today}
\usepackage{lipsum}

% {\tiny word}
% {\scriptsize word}
% {\footnotesize word}
% {\small}
% {\normalsize}
% {\large}
% {\Large}
% {\LARGE}
% {\huge}
% {\Huge}

\begin{document}
	
	\maketitle
	\section{Tiny}
	\paragraph{}
	
	
	{\tiny An ecosystem can be visualised as a functional unit of nature, where living 
		organisms interact among themselves and also with the surrounding physical 
		environment. Ecosystem varies greatly in size from a small pond to a large forest or a 
		sea. Many ecologists regard the entire biosphere as a global ecosystem, as a composite 
		of all local ecosystems on Earth. Since this system is too much big and complex to be 
		studied at one time, it is convenient to divide it into two basic categories, namely the 
		terrestrial and the aquatic. Forest, grassland and desert are some examples of terrestrial 
		ecosystems; pond, lake, wetland, river and estuary are some examples of aquatic 
		ecosystems. Crop fields and an aquarium may also be considered as man-made 
		ecosystems. }
	
	\section{Scriptsize}
	\paragraph{}
	
		{\scriptsize An ecosystem can be visualised as a functional unit of nature, where living 
			organisms interact among themselves and also with the surrounding physical 
			environment. Ecosystem varies greatly in size from a small pond to a large forest or a 
			sea. Many ecologists regard the entire biosphere as a global ecosystem, as a composite 
			of all local ecosystems on Earth. Since this system is too much big and complex to be 
			studied at one time, it is convenient to divide it into two basic categories, namely the 
			terrestrial and the aquatic. Forest, grassland and desert are some examples of terrestrial 
			ecosystems; pond, lake, wetland, river and estuary are some examples of aquatic 
			ecosystems. Crop fields and an aquarium may also be considered as man-made 
			ecosystems. }
	
	\section{footnotesize}
	\paragraph{}
		
		{\footnotesize An ecosystem can be visualised as a functional unit of nature, where living 
			organisms interact among themselves and also with the surrounding physical 
			environment. Ecosystem varies greatly in size from a small pond to a large forest or a 
			sea. Many ecologists regard the entire biosphere as a global ecosystem, as a composite 
			of all local ecosystems on Earth. Since this system is too much big and complex to be 
			studied at one time, it is convenient to divide it into two basic categories, namely the 
			terrestrial and the aquatic. Forest, grassland and desert are some examples of terrestrial 
			ecosystems; pond, lake, wetland, river and estuary are some examples of aquatic 
			ecosystems. Crop fields and an aquarium may also be considered as man-made 
			ecosystems. }
			
	\section{small}
	\paragraph{}
	
	{\small An ecosystem can be visualised as a functional unit of nature, where living 
		organisms interact among themselves and also with the surrounding physical 
		environment. Ecosystem varies greatly in size from a small pond to a large forest or a 
		sea. Many ecologists regard the entire biosphere as a global ecosystem, as a composite 
		of all local ecosystems on Earth. Since this system is too much big and complex to be 
		studied at one time, it is convenient to divide it into two basic categories, namely the 
		terrestrial and the aquatic. Forest, grassland and desert are some examples of terrestrial 
		ecosystems; pond, lake, wetland, river and estuary are some examples of aquatic 
		ecosystems. Crop fields and an aquarium may also be considered as man-made 
		ecosystems. }
	
	\section{normal size}
	\paragraph{}
	
	{\normalsize An ecosystem can be visualised as a functional unit of nature, where living 
		organisms interact among themselves and also with the surrounding physical 
		environment. Ecosystem varies greatly in size from a small pond to a large forest or a 
		sea. Many ecologists regard the entire biosphere as a global ecosystem, as a composite 
		of all local ecosystems on Earth. Since this system is too much big and complex to be 
		studied at one time, it is convenient to divide it into two basic categories, namely the 
		terrestrial and the aquatic. Forest, grassland and desert are some examples of terrestrial 
		ecosystems; pond, lake, wetland, river and estuary are some examples of aquatic 
		ecosystems. Crop fields and an aquarium may also be considered as man-made 
		ecosystems. }
		
	\section{large}
	\paragraph{}
	
	{\large An ecosystem can be visualised as a functional unit of nature, where living 
		organisms interact among themselves and also with the surrounding physical 
		environment. Ecosystem varies greatly in size from a small pond to a large forest or a 
		sea. Many ecologists regard the entire biosphere as a global ecosystem, as a composite 
		of all local ecosystems on Earth. Since this system is too much big and complex to be 
		studied at one time, it is convenient to divide it into two basic categories, namely the 
		terrestrial and the aquatic. Forest, grassland and desert are some examples of terrestrial 
		ecosystems; pond, lake, wetland, river and estuary are some examples of aquatic 
		ecosystems. Crop fields and an aquarium may also be considered as man-made 
		ecosystems. }
	
	\section{Large}
	\paragraph{}
	
	{\Large An ecosystem can be visualised as a functional unit of nature, where living 
		organisms interact among themselves and also with the surrounding physical 
		environment. Ecosystem varies greatly in size from a small pond to a large forest or a 
		sea. Many ecologists regard the entire biosphere as a global ecosystem, as a composite 
		of all local ecosystems on Earth. Since this system is too much big and complex to be 
		studied at one time, it is convenient to divide it into two basic categories, namely the 
		terrestrial and the aquatic. Forest, grassland and desert are some examples of terrestrial 
		ecosystems; pond, lake, wetland, river and estuary are some examples of aquatic 
		ecosystems. Crop fields and an aquarium may also be considered as man-made 
		ecosystems. }
	
	\section{LARGE}
	\paragraph{}
	
	{\LARGE An ecosystem can be visualised as a functional unit of nature, where living 
		organisms interact among themselves and also with the surrounding physical 
		environment. Ecosystem varies greatly in size from a small pond to a large forest or a 
		sea. Many ecologists regard the entire biosphere as a global ecosystem, as a composite 
		of all local ecosystems on Earth. Since this system is too much big and complex to be 
		studied at one time, it is convenient to divide it into two basic categories, namely the 
		terrestrial and the aquatic. Forest, grassland and desert are some examples of terrestrial 
		ecosystems; pond, lake, wetland, river and estuary are some examples of aquatic 
		ecosystems. Crop fields and an aquarium may also be considered as man-made 
		ecosystems. }
	
	\section{huge}
	\paragraph{}
	
	{\huge An ecosystem can be visualised as a functional unit of nature, where living 
		organisms interact among themselves and also with the surrounding physical 
		environment. Ecosystem varies greatly in size from a small pond to a large forest or a 
		sea. Many ecologists regard the entire biosphere as a global ecosystem, as a composite 
		of all local ecosystems on Earth. Since this system is too much big and complex to be 
		studied at one time, it is convenient to divide it into two basic categories, namely the 
		terrestrial and the aquatic. Forest, grassland and desert are some examples of terrestrial 
		ecosystems; pond, lake, wetland, river and estuary are some examples of aquatic 
		ecosystems. Crop fields and an aquarium may also be considered as man-made 
		ecosystems. }
	
	\section{Huge}
	\paragraph{}
	
	{\Huge An ecosystem can be visualised as a functional unit of nature, where living 
		organisms interact among themselves and also with the surrounding physical 
		environment. Ecosystem varies greatly in size from a small pond to a large forest or a 
		sea. Many ecologists regard the entire biosphere as a global ecosystem, as a composite 
		of all local ecosystems on Earth. Since this system is too much big and complex to be 
		studied at one time, it is convenient to divide it into two basic categories, namely the 
		terrestrial and the aquatic. Forest, grassland and desert are some examples of terrestrial 
		ecosystems; pond, lake, wetland, river and estuary are some examples of aquatic 
		ecosystems. Crop fields and an aquarium may also be considered as man-made 
		ecosystems. }
		
\end{document}