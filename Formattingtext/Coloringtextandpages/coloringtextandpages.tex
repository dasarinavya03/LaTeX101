\documentclass[12pt]{article}
\usepackage[none]{hyphenat}
\usepackage[a4paper, margin=1.5in]{geometry}
\usepackage[dvipsnames]{xcolor}
\title{Coloring Text and Pages}
\author{Navya Dasari}
\date{\today}
\usepackage{lipsum}

\begin{document}
	
	\maketitle
	\section{Single Colour}
	\paragraph{}
	
	\textcolor{blue}{An ecosystem can be visualised as a functional unit of nature, where living 
		organisms interact among themselves and also with the surrounding physical 
		environment. Ecosystem varies greatly in size from a small pond to a large forest or a 
		sea. Many ecologists regard the entire biosphere as a global ecosystem, as a composite 
		of all local ecosystems on Earth. Since this system is too much big and complex to be 
		studied at one time, it is convenient to divide it into two basic categories, namely the 
		terrestrial and the aquatic. Forest, grassland and desert are some examples of terrestrial 
		ecosystems; pond, lake, wetland, river and estuary are some examples of aquatic 
		ecosystems. Crop fields and an aquarium may also be considered as man-made 
		ecosystems.} \\
		
		\section{Mixture of Single Colour and White}
		
	\textcolor{red!70}{An ecosystem can be visualised as a functional unit of nature, where living 
		organisms interact among themselves and also with the surrounding physical 
		environment. Ecosystem varies greatly in size from a small pond to a large forest or a 
		sea. Many ecologists regard the entire biosphere as a global ecosystem, as a composite 
		of all local ecosystems on Earth. Since this system is too much big and complex to be 
		studied at one time, it is convenient to divide it into two basic categories, namely the 
		terrestrial and the aquatic. Forest, grassland and desert are some examples of terrestrial 
		ecosystems; pond, lake, wetland, river and estuary are some examples of aquatic 
		ecosystems. Crop fields and an aquarium may also be considered as man-made 
		ecosystems.} \\
		
		\section{Mixture of Two Colors}
		
	\textcolor{red!70!blue}{An ecosystem can be visualised as a functional unit of nature, where living 
		organisms interact among themselves and also with the surrounding physical 
		environment. Ecosystem varies greatly in size from a small pond to a large forest or a 
		sea. Many ecologists regard the entire biosphere as a global ecosystem, as a composite 
		of all local ecosystems on Earth. Since this system is too much big and complex to be 
		studied at one time, it is convenient to divide it into two basic categories, namely the 
		terrestrial and the aquatic. Forest, grassland and desert are some examples of terrestrial 
		ecosystems; pond, lake, wetland, river and estuary are some examples of aquatic 
		ecosystems. Crop fields and an aquarium may also be considered as man-made 
		ecosystems.}
	
	\newpage
	
	\section{Page Color}
	
	\pagecolor{Goldenrod! 50}
	
	An ecosystem can be visualised as a functional unit of nature, where living 
	organisms interact among themselves and also with the surrounding physical 
	environment. Ecosystem varies greatly in size from a small pond to a large forest or a 
	sea. Many ecologists regard the entire biosphere as a global ecosystem, as a composite 
	of all local ecosystems on Earth. Since this system is too much big and complex to be 
	studied at one time, it is convenient to divide it into two basic categories, namely the 
	terrestrial and the aquatic. Forest, grassland and desert are some examples of terrestrial 
	ecosystems; pond, lake, wetland, river and estuary are some examples of aquatic 
	ecosystems. Crop fields and an aquarium may also be considered as man-made 
	ecosystems.
	
	
	
\end{document}