\documentclass[12pt]{article}
\usepackage{amsmath}
\usepackage[none]{hyphenat}
\usepackage[a4paper, margin=1.5in]{geometry}
\title{Mathematics}
\author{Navya Dasari}
\date{\today}
\usepackage{lipsum}

\begin{document}
	\maketitle
	
	\section{Simple Mathematics}
	
	\paragraph{}
	
	The basic mathematical equations are written here using LaTeX: \\
	
	$2^2 + 2^2 = 8$ \\
	
	$\sqrt[4]{4096} = 8$ \\
	
	$e^{x+iy} = e^x(\cos y + i\sin y)$ \\
	
	$A \cup B = n(A) + n(B) - n(A \cap B)$ \\
	
	$\cos^2\theta + \sin^2\theta = 1$  \\
	
	$A \cup B = \{x \in A \ {or} \ x \in B\}$
	
	\section{Fractions}
	
	\paragraph{}
	
	Simple Fractions are written here in LaTeX: \\
	
	$fraction = \frac{numerator}{denominator}$  \\
	
	$\frac{2}{3}$  \\
	
	$\frac{8}{\frac{8}{3}}$ \\
	
	$\frac{a}{b} \geq \frac{c}{d}$  \\
	
	$\frac{\sqrt{x + 2}}{x^2 - 3}$  \\
	
	$\frac{a}{\frac{b}{c}} \times \frac{\frac{d}{e}}{f} \geq 1$
	
	\section{Variable Size of Braces}
	
	% \Bigg , \bigg , \Big , $\big$
	
	\paragraph{}
	
	Here, Different sizes of braces are used to write the equation in LaTeX: \\
	
	$ \bigg[\Big\{\big(7+3\big)/5 \Big\} \times 6 \bigg]$ \\[12pt]
	
	
	$\bigg\{\Big(\frac{8}{4} \Big) + \Big(\frac{10}{2}\Big)\bigg\}$ \\
		
	\section{Summation}
	
	\paragraph{}
	
	How to write summation in LaTeX??
	
	\[\sum_{i=a}^{b}g(i) = 0, {for} \ b < a\] \\
	
	\[\sum_{i=1}^{n}i = \frac{n(n+1)}{2}\] \\
	
	\section{Integration and Limits}
	
	\[\int_{0}^{\infty}f(x)dx\] \\
	
	\[\lim_{x \to c} f(x) = L\]  \\
	
	\newpage
	
	\section{Matrix}
	
	 	% matrix: without border
	 	% pmatrix : with round border
	 	% bmatrix : with box bracket
	 	% Bmatrix : with curly bracket
	 	% vmatrix : with |
	 	% Vmatrix : with ||
	 	% & denotes the new column
	 	% \\ denotes a new row
	\paragraph{}
	
	Matrix written in LaTeX: \\
	
	$
	\begin{pmatrix}
		20 & 40 & 50 \\
		34 & 48 & 60 \\
		40 & 50 & 60 \\
	\end{pmatrix}
	$	\\[12pt]
	
	$
	\begin{bmatrix}
		20 & 40 & 50 \\
		34 & 48 & 60 \\
		40 & 50 & 60 \\
	\end{bmatrix}
	$ \\[12pt]
	
	\section{Addition of Matrix}
	
	\paragraph{}
	
	Matrix Addition is shown here: \\
	
	$
	\begin{pmatrix}
		1 & 2 \\
		3 & 4 \\
	\end{pmatrix}
	+
	\begin{pmatrix}
		5 & 6 \\
		7 & 8 \\
	\end{pmatrix}
	=
	\begin{pmatrix}
		6 & 8 \\
		10 & 12 \\
	\end{pmatrix}
	$ \\[12pt]
	
	$
	\begin{bmatrix}
		1 & 2 \\
		3 & 4 \\
	\end{bmatrix}
	+
	\begin{bmatrix}
		5 & 6 \\
		7 & 8 \\
	\end{bmatrix}
	=
	\begin{bmatrix}
		6 & 8 \\
		10 & 12 \\
	\end{bmatrix}
	$
	
	\section{Equation}
	
	\paragraph{}
	
	In this section, it is shown how equations are centered and numbered in LaTeX.
	
	\begin{equation}
		3x + 4y = 2
	\end{equation}
	
	\begin{equation}
		15x + 25y = 40
	\end{equation}
	
	\begin{equation}
		x^{2} - y^{2} = (x+y)(x-y)
	\end{equation}
	
	\section{Equation Align Environment}
	
	\begin{align}
		3x - 6 = 9  \\
		3x& = 9 + 6 \nonumber \\
		x& = \frac{9+6}{3} \nonumber \\
		x& = 5 \nonumber
	\end{align}
	
\end{document}